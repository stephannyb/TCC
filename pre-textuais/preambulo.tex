%
% ********** Página de assinaturas
%

\begin{titlepage}

\begin{center}

\LARGE

\textbf{MarIA - Um chatbot inteligente para atendimento de vítimas de violência contra a mulher}

\vfill

\Large

\textbf{Tereza Stephanny de Brito Félix}

\end{center}

\vfill

% O \noindent é para eliminar a tabulação inicial que normalmente é
% colocada na primeira frase dos parágrafos
\noindent
% Descomente a opção que se aplica ao seu caso
% Note que propostas de tema de qualificação nunca têm preâmbulo.
Monografia aprovada em \today, pela banca examinadora composta
pelos seguintes membros:

% Os nomes dos membros da banca examinadora devem ser listados
% na seguinte ordem: orientador, co-orientador (caso haja),
% examinadores externos, examinadores internos. Dentro de uma mesma
% categoria, por ordem alfabética

\begin{center}

\vspace{1.5cm}\rule{0.95\linewidth}{1pt}
\parbox{0.9\linewidth}{%
Prof. Dr. Carlos Manuel Dias Viegas (orientador) \dotfill\ DCA/UFRN}

\vspace{1.5cm}\rule{0.95\linewidth}{1pt}
\parbox{0.9\linewidth}{%
Prof. Dr. YYYYY (co-orientador) \dotfill\ MCA/UFRN}

\vspace{1.5cm}\rule{0.95\linewidth}{1pt}
\parbox{0.9\linewidth}{%
Prof. Dr. WWWWWW \dotfill\ DEM/UFFN}

\vspace{1.5cm}\rule{0.95\linewidth}{1pt}
\parbox{0.9\linewidth}{%
Profª Drª ZZZZZZ \dotfill\ DEE/UFRN}

\end{center}

\end{titlepage}

%
% ********** Dedicatória
%

% A dedicatória não é obrigatória. Se você tem alguém ou algo que teve
% uma importância fundamental ao longo do seu curso, pode dedicar a ele(a)
% este trabalho. Geralmente não se faz dedicatória a várias pessoas: para
% isso existe a seção de agradecimentos.
% Se não quiser dedicatória, basta excluir o texto entre
% \begin{titlepage} e \end{titlepage}

\begin{titlepage}

\vspace*{\fill}

\hfill
\begin{minipage}{0.5\linewidth}
\begin{flushright}
\large\it
Ao meu filho, Pedro Gabriel de Brito Félix Gomes, que  todos os dias me inspira a ser uma pessoa melhor.
\end{flushright}
\end{minipage}

\vspace*{\fill}

\end{titlepage}

%
% ********** Agradecimentos
%

% Os agradecimentos não são obrigatórios. Se existem pessoas que lhe
% ajudaram ao longo do seu curso, pode incluir um agradecimento.
% Se não quiser agradecimentos, basta excluir o texto após \chapter*{...}

\chapter*{Agradecimentos}
\thispagestyle{empty}

\begin{trivlist}  \itemsep 2ex
\item Agradeço em primeiro lugar a Deus por sua infinita bondade e misericórdia, por me sustentar em todos os dias da minha vida. À Imaculada Conceição e a Sant'Ana por suas interseções e proteção.

\item Agradeço ao meu orientador por sua paciência e disponibilidade. 

\item Agradeço ao meu filho, Pedro Gabriel, que sempre me viu como "super", que acreditou em mim quando eu mesma não acreditava.

\item Agradeço ao meu amor, Pedro Augusto, pelo apoio incondicional, pelo cuidado, pelo consolo e por me mostrar com tanto amor a capacidade em mim que eu custei a enxergar. 

\item Aos meus amigos pelo cuidado, em especial Clausan Liano e Jemima Feitosa, com quem eu tive a honra e alegria de dividir meus dias de trabalho e por sempre "viajarem" nas minhas ideias comigo. Aos demais, pelas críticas e sugestões.

\item À minha psicóloga, Thayse Lira, por todas as horas de conversas, explicações, orientações e cuidado.

\item Aos meus pais, Adenilza Jerônimo de Brito e Rogério das Chagas de Oliveira Félix, que não mediriam esforços para que a menininha deles, lá do interior, chegasse até aqui. E a minha família pelo apoio durante esta jornada.

\item A todos que de alguma forma contribuíram para este trabalho fosse possível, muito obrigada.




\end{trivlist}
