%
% ********** Resumo
%

% Usa-se \chapter*, e não \chapter, porque este "capítulo" não deve
% ser numerado.
% Na maioria das vezes, ao invés dos comandos LaTeX \chapter e \chapter*,
% deve-se usar as nossas versões definidas no arquivo comandos.tex,
% \mychapter e \mychapterast. Isto porque os comandos LaTeX têm um erro
% que faz com que eles sempre coloquem o número da página no rodapé na
% primeira página do capítulo, mesmo que o estilo que estejamos usando
% para numeração seja outro.
\mychapterast{Resumo}

A violência contra a mulher é um problema estrutural que demanda soluções tecnológicas para facilitar o acesso das vítimas aos serviços de emergência. Este trabalho apresenta o desenvolvimento de um chatbot para o Telegram, utilizando Modelos de Linguagem de Grande Escala (LLM - Large Language Model) e técnicas de Inteligência Artificial (IA), para classificação e atendimento automatizado de ocorrências de violência contra a mulher. A solução foi projetada para permitir que vítimas entrem em contato de maneira discreta e segura, utilizando mensagens de texto, áudio, imagens, vídeos e localização. O chatbot processa as informações recebidas, classifica a gravidade da ocorrência e direciona os dados para os órgãos competentes, agilizando o atendimento. Além da implementação do sistema, este estudo discute os desafios técnicos, éticos e jurídicos envolvidos na aplicação da IA para segurança pública, destacando os impactos sociais da tecnologia no combate à violência de gênero.

\vspace{1.5ex}

{\bf Palavras-chave}: Chatbot; Inteligência Artificial; Violência contra a Mulher; Linguagem Natural; Atendimento Emergencial.

%
% ********** Abstract
%
\mychapterast{Abstract}

Violence against women is a structural problem that requires technological solutions to facilitate victims' access to emergency services. This work presents the development of a Telegram chatbot, utilizing Large Language Models (LLMs) and Artificial Intelligence (AI) techniques for automated classification and response to incidents of violence against women. The solution was designed to allow victims to contact emergency services in a discreet and secure manner, using text messages, audio, images, videos, and location data. The chatbot processes the received information, classifies the severity of the incident, and forwards the data to the appropriate authorities, streamlining the response process. In addition to the system implementation, this study discusses the technical, ethical, and legal challenges involved in applying AI to public safety, highlighting the social impact of technology in combating gender-based violence.

\vspace{1.5ex}

{\bf Keywords}: Chatbot; Artificial Intelligence; Violence Against Woman, Natural Language Processing; Emergency Response.
