%
%% Capítulo Introducao
%%
\mychapter{INTRODUÇÃO}\label{Cap:Introducao}

Este projeto descreve o desenvolvimento de um \textit{chatbot} inteligente no \textit{Telegram}, que integra Modelos de Linguagem de Grande Escala (\textit{Large Language Model} \- LLM\nomenclature{LLM}{Large Language Model}). A ferramenta que tratamos neste trabalho foi projetada especificamente para oferecer acolhimento e facilitar o registro de ocorrências de violência doméstica contra a mulher, permitindo que as vítimas relatem casos com maior segurança, sigilo e facilidade, superando algumas barreiras iniciais no acesso à justiça.

Para a clareza da exposição, é fundamental estabelecer duas definições centrais a este trabalho. Primeiramente, entende-se por \textit{chatbot} um programa de computador que simula a conversação humana para automatizar tarefas e fornecer informações. Em segundo lugar, a violência contra a mulher é compreendida como qualquer ato baseado no gênero que resulte em dano físico, sexual ou psicológico, representando uma grave violação dos direitos humanos.

\begin{figure}[ht]
	\includegraphics[width=0.4\textwidth, keepaspectratio=true]{figuras/Intro/diagrama.png}
	\centering
	\caption[Diagrama Preliminar]{Diagrama Preliminar.}
	\footnotesize{Fonte: Produção da Autora.}\label{fig:diagrama.png}
\end{figure}
\FloatBarrier%

Além de servir como um canal de comunicação mais acessível para contatar os órgãos de segurança pública, conforme ilustrado na~\autoref{fig:diagrama.png}, o sistema também poderá se tornar uma fonte de dados, os quais podem subsidiar a adaptação e o desenvolvimento de políticas públicas mais eficazes no combate à violência contra a mulher.

\section{VIOLÊNCIA CONTRA A MULHER} 

A violência contra a mulher é um fenômeno complexo e multifacetado que atinge diversas esferas da sociedade, sendo o Brasil regido pela Lei Federal nº 11.340, de 7 de Agosto de 2006, conhecida como Lei Maria da Penha. Esta legislação define a violência doméstica e familiar contra a mulher como qualquer ação ou omissão baseada no gênero que lhe cause morte, lesão, sofrimento físico, sexual ou psicológico, além de dano moral ou patrimonial \cite{LeiMariaDaPenha}.

Em sua abrangência, a lei detalha as cinco formas principais de agressão que, frequentemente, se interrelacionam e coexistem no cotidiano das vítimas. Estas abrangem a violência física, que ofende a integridade ou saúde corporal, a violência psicológica, que causa dano emocional, diminuição da autoestima, controle ou isolamento, a violência sexual, que a força a participar de relação não desejada, a violência patrimonial, retenção ou destruição de bens, e a violência moral, que se da por meio de calúnia, difamação ou injúria. Os impactos dessas agressões são profundos, afetando a saúde física, mental, a autonomia e o bem-estar das mulheres.

\section{MOTIVAÇÃO}
A violência contra a mulher é uma das mais graves violações aos direitos humanos no Brasil. De acordo com o Anuário Brasileiro de Segurança Pública~\cite{forum2023}, cresceram todas as modalidades de violência contra a mulher. E segundo o painel estatístico do Ministério da Justiça e Segurança Pública, durante o ano de 2024 foram registradas 1.510 mortes de mulheres em decorrência de feminicídios~\cite{mj2024painel}. Apesar do aumento de campanhas de conscientização, muitas vítimas ainda enfrentam obstáculos para denunciar seus agressores \- entre eles, o medo, a vergonha, a dependência econômica ou emocional e a dificuldade de acesso aos meios de denúncia convencionais.

Esse cenário revela a urgência por soluções tecnológicas inovadoras, que ofereçam segurança, orientação e acolhimento às vítimas. O presente projeto propõe o uso de um \textit{chatbot} inteligente baseado em Modelos de Linguagem de Grande Escala, como o GPT-4, capaz de receber informações, identificar e classificar automaticamente ocorrências de violência conforme os tipos previstos na Lei Maria da Penha~\cite{LeiMariaDaPenha}.

\section{OBJETIVOS}
Este trabalho tem como objetivo desenvolver um sistema baseado em Inteligência Artificial (IA) para a captação, categorização e encaminhamento de ocorrências de violência doméstica contra a mulher, com foco no acolhimento inicial, na orientação das vítimas e na organização de dados úteis para órgãos públicos e entidades da sociedade civil.

A solução será implementada por meio de um \textit{chatbot} acessível pelo aplicativo \textit{Telegram}, visando garantir usabilidade e segurança para as usuárias. E contará com uma Interface de Programação de Aplicação (\textit{Application Programming Interface} \- API\nomenclature{API}{Application Programming Interface}) integrada a modelos de linguagem natural, responsável por interpretar as mensagens, classificar os relatos e encaminhá-los de forma adequada. Conforme ilustrado na~\autoref{fig:2.png}. 

\subsection{Objetivos Específicos}

Para alcançar o objetivo geral, foram definidos três objetivos específicos interligados. O primeiro consiste em desenvolver um \textit{chatbot} integrado ao aplicativo \textit{Telegram}, projetado para captar relatos de violência doméstica de forma segura e acessível, garantindo a privacidade das vítimas durante todo o processo de comunicação.

O segundo objetivo é integrar esse \textit{chatbot} a uma API equipada com Inteligência Artificial, baseada em LLM.\@ Essa integração tem a finalidade de interpretar e categorizar os relatos recebidos, além de oferecer orientações iniciais às vítimas e direcioná-las adequadamente aos órgãos competentes.

Por fim, o terceiro objetivo visa organizar e estruturar os dados obtidos de modo a torná-los acessíveis e úteis para órgãos de segurança pública, organizações não governamentais e instituições de apoio. Essa estruturação busca contribuir para a formulação e o aprimoramento de políticas públicas de enfrentamento à violência de gênero, em conformidade com a Lei Maria da Penha e demais diretrizes legais vigentes.

\begin{figure}[ht]
	\includegraphics[width=0.4\textwidth, keepaspectratio=true]{figuras/Intro/2.png}
	\centering
	\caption[Diagrama de Fluxo]{Diagrama de Fluxo.}
	\footnotesize{Fonte: Produção da Autora.}\label{fig:2.png}
\end{figure}
\FloatBarrier%

\section{TRABALHOS RELACIONADOS}

Nos últimos anos, diversas soluções tecnológicas têm sido desenvolvidas com o objetivo de apoiar mulheres em situação de violência. Dentre essas iniciativas, destacam-se o aplicativo PenhaS, criado pelo Instituto AzMina, que conecta usuárias a serviços de apoio e fornece informações sobre seus direitos \cite{azmina2024}; o \textit{Maria da Penha Virtual}, do Tribunal de Justiça do Estado do Rio de Janeiro, que permite o pedido online de medidas protetivas \cite{tjrj2021}; e o aplicativo Salve Elas, desenvolvido por meio de parceria entre o Instituto Metrópole Digital (IMD/UFRN) e o Governo do Rio Grande do Norte, que permite que as mulheres com medida protetiva deferida realizem o acionamento discreto da Polícia Militar via integração com o Centro Integrado de Operações de Segurança Pública (CIOSP), indicando localização geográfica da usuária em situação de risco \cite{salveelas2023}.

Essas ferramentas representam avanços significativos na democratização da informação e no acesso à rede de proteção.

No entanto, estudos apontam que muitas dessas soluções ainda enfrentam limitações quanto à empatia no atendimento, usabilidade e integração com instituições públicas \cite{caldeira2024}. Essas lacunas evidenciam a necessidade de sistemas mais inteligentes, acolhedores e sensíveis ao contexto das usuárias, o que reforça a importância do desenvolvimento de um \textit{chatbot} com abordagem centrada na vítima, como o proposto neste trabalho.

\subsection{LIMITAÇÕES DAS SOLUÇÕES EXISTENTES}
A análise desses sistemas revela limitações que comprometem sua eficácia, tais como:  

\begin{itemize}
    \item interfaces pouco intuitivas para usuárias em situação de estresse;  
    \item comunicação rígida e pouco empática, baseada em fluxos fixos de perguntas e respostas;  
    \item ausência de integração completa com órgãos públicos de segurança e acolhimento;  
    \item baixa confiabilidade dos dados coletados devido à interrupção frequente dos relatos.  
\end{itemize}

Essas fragilidades apontam para um descompasso entre as potencialidades das tecnologias de IA e as necessidades reais das mulheres que recorrem a tais ferramentas em momentos críticos.

\section{PROBLEMA}

Nesta seção foram apresentados dados que evidenciam a persistência e gravidade da violência contra a mulher no Brasil, incluindo os altos índices de feminicídio e a subnotificação dos casos. Apesar da existência de legislações específicas, como a Lei Maria da Penha, e da crescente adoção de tecnologias de apoio, as vítimas ainda enfrentam barreiras para acessar canais de denúncia de forma rápida, acolhedora e segura.  

Esse cenário reforça a necessidade de explorar soluções inovadoras baseadas em tecnologias emergentes como \textit{chatbots} e inteligência artificial aplicada a contextos sociais e sensíveis.

\section{ESTRUTURA DO TRABALHO}
Este trabalho está estruturado em capítulos, sendo o primeiro uma introdução sobre o tema, mostrando os fatores que motivam a implantação da ideia, além da justificativa e dos objetivos. Em sequência, o capítulo~\ref{Cap:Teoria} aborda o referencial teórico. O capítulo~\ref{Cap:Implementacao}, por sua vez, explica a metodologia para desenvolvimento do \textit{chatbot} inteligente, enquanto o capítulo~\ref{Cap:ExperimentosResultados} trata dos resultados. Por fim, o capítulo~\ref{Cap:Conclusao} traz as principais conclusões e contribuições deste trabalho.