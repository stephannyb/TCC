%%
%% Capítulo 4: Problema
%%

\mychapter{PROBLEMA}
\label{Cap:Problema}

\section{CONTEXTUALIZAÇÃO DO PROBLEMA}
Na Introdução (Capítulo~\ref{Cap:Introducao}), foram apresentados dados que evidenciam a persistência e gravidade da violência contra a mulher no Brasil, incluindo os altos índices de feminicídio e a subnotificação dos casos. Apesar da existência de legislações específicas, como a Lei Maria da Penha, e da crescente adoção de tecnologias de apoio, as vítimas ainda enfrentam barreiras para acessar canais de denúncia de forma rápida, acolhedora e segura.  

Esse cenário reforça a necessidade de explorar soluções inovadoras baseadas em tecnologias emergentes, como discutido na Fundamentação Teórica (Capítulo~\ref{Cap:Teoria}), que abordou os avanços em processamento de linguagem natural, \textit{chatbots} e inteligência artificial aplicada a contextos sociais e sensíveis.  

\section{LIMITAÇÕES DAS SOLUÇÕES EXISTENTES}
Conforme descrito em Trabalhos Relacionados (Capítulo~\ref{Cap:TrabalhosRelacionados}), já existem iniciativas relevantes, como o PenhaS, o Maria da Penha Virtual e o \textit{chatbot} Aurora. No entanto, a análise desses sistemas revela limitações que comprometem sua eficácia, tais como:  

\begin{itemize}
    \item interfaces pouco intuitivas para usuárias em situação de estresse;  
    \item comunicação rígida e pouco empática, baseada em fluxos fixos de perguntas e respostas;  
    \item ausência de integração completa com órgãos públicos de segurança e acolhimento;  
    \item baixa confiabilidade dos dados coletados devido à interrupção frequente dos relatos.  
\end{itemize}

Essas fragilidades apontam para um descompasso entre as potencialidades das tecnologias de IA e as necessidades reais das mulheres que recorrem a tais ferramentas em momentos críticos.  

\section{FORMULAÇÃO DO PROBLEMA DE PESQUISA}
A partir dessa análise, torna-se evidente que as soluções atuais não conseguem equilibrar os aspectos técnicos, sociais e humanos envolvidos no atendimento de ocorrências de violência contra a mulher.  

Assim, o problema de pesquisa que orienta este trabalho pode ser formulado da seguinte forma:  

\begin{quote}
Como desenvolver um \textit{chatbot} inteligente, capaz de realizar atendimento empático, seguro e tecnicamente viável, de modo a acolher e encaminhar ocorrências de violência contra a mulher, superando as limitações identificadas nas soluções atuais?
\end{quote}

\section{HIPÓTESE}
Considera-se como hipótese que a integração de modelos de linguagem natural baseados em LLMs, aliados a um fluxo estruturado de coleta de informações e mecanismos de armazenamento seguro, pode oferecer uma experiência de atendimento mais acolhedora. Essa abordagem tem o potencial de aumentar a adesão das usuárias, reduzir interrupções durante o relato e melhorar a qualidade dos dados encaminhados aos órgãos competentes.  
