%%
%% Capítulo 7: Conclusões
%%

\mychapter{CONCLUSÃO}\label{Cap:Conclusao}

Este trabalho buscou responder ao problema de como desenvolver uma ferramenta tecnológica capaz de oferecer atendimento inicial empático, seguro e eficiente a vítimas de violência contra a mulher, superando limitações de interfaces rígidas e pouco acolhedoras. A proposta central foi o desenvolvimento do MarIA, um \textit{chatbot} inteligente integrado ao \textit{Telegram}, que utiliza Modelos de Linguagem de Grande Escala (LLM) para interpretar relatos e estruturar ocorrências.

\section{RESPOSTA AO PROBLEMA DE PESQUISA}

Os resultados obtidos em ambiente controlado demonstram a viabilidade técnica e funcional da solução proposta. O sistema foi capaz de realizar atendimentos simulados de forma humanizada, interpretando linguagem natural e aceitando múltiplas formas de entrada (texto, áudio, imagem), o que reduz barreiras de comunicação em momentos de estresse. A integração com LLMs permitiu a classificação automática dos tipos de violência, agilizando o processo de triagem sem perder a sensibilidade necessária ao tema.

Portanto, a hipótese inicial de que a união entre interfaces conversacionais acessíveis e inteligência artificial generativa pode melhorar a experiência de denúncia foi corroborada pelos testes funcionais. A solução mostrou-se tecnicamente viável e com potencial para preencher as lacunas identificadas nas ferramentas atualmente disponíveis.

\section{CONTRIBUIÇÕES}

As principais contribuições deste estudo incluem a validação de uma arquitetura de microsserviços para sistemas sensíveis, garantindo modularidade e segurança no tratamento de dados. Além disso, o trabalho entregou um protótipo funcional que não apenas coleta relatos, mas os estrutura de forma útil para a formulação de políticas públicas, como demonstrado pelos painéis de monitoramento desenvolvidos. O projeto também contribui com a discussão sobre o uso ético de IA em contextos sociais críticos.

\section{LIMITAÇÕES}

É importante ressaltar que a validação ocorreu em cenários simulados, não tendo sido o sistema submetido ao imprevisível ambiente de produção real com vítimas reais. Limitações técnicas também foram observadas, como a dependência da qualidade de conexão para a transcrição de áudios em tempo real e a precisão variável de modelos de OCR em imagens de baixa resolução. Além disso, a eficácia do acolhimento ``empático '' por uma IA, embora promissora nos testes, carece de validação com usuárias reais e acompanhamento psicológico.

\section{TRABALHOS FUTUROS}

Para trabalhos futuros, sugere-se a realização de um estudo piloto assistido por profissionais da rede de proteção à mulher, a fim de validar a eficácia do acolhimento em situações reais. Tecnicamente, recomenda-se aprimorar os modelos de reconhecimento de voz e imagem para funcionarem melhor em condições adversas. A expansão para outras plataformas de mensageria amplamente utilizadas, como o \textit{WhatsApp}, também representa um passo natural para aumentar o alcance da ferramenta.

\section{CONSIDERAÇÕES FINAIS}

O MarIA representa um passo inicial, porém firme, na direção de usar tecnologia de ponta para combater uma das violações de direitos humanos mais persistentes. Este trabalho evidenciou que a tecnologia, quando projetada com empatia e responsabilidade, pode ser uma aliada poderosa na proteção e no acolhimento de mulheres, oferecendo um canal de escuta que está sempre disponível.