%%
%% Capítulo 3: Trabalhos Relacionados
%%

\mychapter{TRABALHOS RELACIONADOS}
\label{Cap:TrabalhosRelacionados}

Nos últimos anos, diversas soluções tecnológicas têm sido desenvolvidas com o objetivo de apoiar mulheres em situação de violência. Dentre essas iniciativas, destacam-se o aplicativo PenhaS, criado pelo Instituto AzMina, que conecta usuárias a serviços de apoio e fornece informações sobre seus direitos \cite{azmina2024}; o \textit{Maria da Penha Virtual}, do Tribunal de Justiça do Estado do Rio de Janeiro, que permite o pedido online de medidas protetivas \cite{tjrj2021}; e o aplicativo Salve Elas, desenvolvido por meio de parceria entre o Instituto Metrópole Digital (IMD/UFRN) e o Governo do Rio Grande do Norte, que permite que as mulheres com medida protetiva deferida realizem o acionamento discreto da Polícia Militar via integração com o Centro Integrado de Operações de Segurança Pública (CIOSP), indicando localização geográfica da usuária em situação de risco \cite{salveelas2023}.

Essas ferramentas representam avanços significativos na democratização da informação e no acesso à rede de proteção.

No entanto, estudos apontam que muitas dessas soluções ainda enfrentam limitações quanto à empatia no atendimento, usabilidade e integração com instituições públicas \cite{caldeira2024}. Essas lacunas evidenciam a necessidade de sistemas mais inteligentes, acolhedores e sensíveis ao contexto das usuárias, o que reforça a importância do desenvolvimento de um \textit{chatbot} com abordagem centrada na vítima, como o proposto neste trabalho.